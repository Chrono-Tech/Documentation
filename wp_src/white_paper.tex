\documentclass[9pt,oneside,reqno]{amsart}

\usepackage{amsmath}
\usepackage{amssymb}
\usepackage{caption}
\usepackage{multicol}
%\usepackage{natbib}
\usepackage{float}
\usepackage[a4paper,width=170mm,top=18mm,bottom=22mm,includeheadfoot]{geometry}
\usepackage{graphicx}% Include figure files
\usepackage{dcolumn}% Align table columns on decimal point
\usepackage{bm}% bold math
\usepackage{xcolor}
\usepackage[colorlinks=true,linkcolor={blue!50!black},citecolor={blue!50!black},urlcolor={blue!80!black}]{hyperref}
\usepackage{mathtools} % For the Ceil and Floor functions
\usepackage{siunitx}
%\setlength{\parindent}{0pt} % Remove paragraph indentation
\usepackage{lineno}

% Custom Ceil and Floor Commands
\DeclarePairedDelimiter{\ceil}{\lceil}{\rceil}
\DeclarePairedDelimiter{\floor}{\lfloor}{\rfloor}

% Change section headings
\makeatletter
\renewcommand\section{\@startsection{section}{1}%
\z@{1.5\linespacing\@plus\linespacing}{\linespacing}%
{\normalfont\Large\scshape\centering}}
\renewcommand\subsection{\@startsection{subsection}{2}%
\z@{1.5\linespacing}{.5\linespacing}%
{\normalfont\scshape\centering\large}}
\renewcommand\subsubsection{\@startsection{subsubsection}{3}%
\z@{.5\linespacing\@plus.7\linespacing}{0.5\linespacing}%
{\normalfont\scshape\centering}}
\makeatother

% Drafting Tools
%\linenumbers
%\usepackage{draftwatermark}
%\SetWatermarkText{DRAFT}
%\SetWatermarkScale{8}
%\SetWatermarkColor[rgb]{1,0.8,0.8}


\captionsetup{width=\columnwidth}

% ======================
% Insert TODO boxes, turn them on or off
% ======================

\newif\ifstatus
\statustrue
%\statusfalse

\newcommand{\todo}[1]{%
\ifstatus
  \begin{center}%
    \fbox{%
      \begin{minipage}{0.95\columnwidth}\small
        \textbf{TODO:} #1
      \end{minipage}%
    }
  \end{center}
\fi
}


% ======================
% make things a bit prettier with microtype...
% ======================

\usepackage{microtype}
\microtypesetup{expansion=alltext,step=1} % hence we can activate expansion ...
\tolerance=300         % ... and make TeX almost as intolerant as it normally is

% ======================


\begin{document}

\title{ChronoBank - Phase 1: A non-volatile digital token backed by labour-hours}

\author{The ChronoBank Team \\
ChronoBank.io \\
info@chronobank.io
}


\begin{abstract} \small
This whitepaper abstractly describes a system designed to tokenise labour-hours using blockchain technology. ChronoBank is a proposed implementation of the described system that can be deployed in several economic localities. The proposed system leverages smart contract techniques to automate a process whereby a country-specific `labour-hour' token may be redeemed for real labour-hours via legally binding (traditional) contracts with labour-offering companies. The proposed `stable-coin' implementation provides a non-volatile, inflation-resistant digital asset transfer system.
\end{abstract}

\maketitle

\setlength{\columnsep}{20pt}
\begin{multicols}{2}

\section{Introduction}

With the advent of cryptocurrencies, relatively instant low-cost transfers of value have become a reality. Blockchain technology, which is a defining feature of most cryptocurrencies, has recently been applied to solve a great variety of problems. Currently the most widespread implementation of blockchain technology is Bitcoin \cite{Nakamoto2008}, which is a simple asset transfer system. The asset in Bitcoin's case is a bitcoin (BTC). The value of this token has seen rapid variation since its inception in 2009, which has hindered its feasibility as a global currency.

There have been a variety of attempts to realise the advantages of blockchain technology while simultaneously mitigating issues regarding the stability of value for cryptocurrency applications. To achieve this, many attempts employ the notion of a \textit{stable-coin}, whereby each token of value in the system has a counterpart of equal worth stored in a non-digital and tangible form in the `real world'.

Two example implementations of the aforementioned \textit{stable-coin} paradigm are listed below:
\begin{description}
	\item[{\bf USDT by Tether\cite{Tether.to2014}}] \hfill \\
		Each USDT token is backed by an equivalent amount of United States Dollars (USD) held in a reserve account by the private company Tether Limited.
	\item[{\bf Digix\cite{Djie2016}}] \hfill \\
		Each token is backed by an equivalent amount of gold, which is stored in reserves by a dedicated precious metal storage custodian.  
\end{description}

In both examples, it is always possible for a token holder to redeem that token for its counterpart, thus ensuring its fundamental `stable' value.

Another notable example of a \textit{stable-coin} is Bitshares \cite{Larimer2015}, which attempts to decentralise the entire system through the use of digital Contract For Differences (CFD) \cite{CFD} interactions. The system presented in this whitepaper does not attempt to achieve decentralisation, but instead attempts to address some of the drawbacks surrounding existing centralised \textit{stable-coins}. These drawbacks include difficulties regarding the storage of physical or economic wealth, and the increasing likelihood of attacks, as a single entity centralises the entire wealth of the system. Typical \textit{stable-coins} are also subject to fluctuations in the value of their underlying asset. While these fluctuations are usually very small when compared with fluctuations in traditional cryptocurrencies, they are still significant. For example, USDT is subject to the devaluation of USD due to inflation.

In this paper, we propose a stable cryptocurrency system which addresses the aforementioned drawbacks of existing stable currency solutions. Specifically, we propose a new type of token which is not backed by any existing fiat currency or physical store of wealth, but instead is backed by legally binding contractual obligations to provide real-world labour-hours. As such, the system and its controlling entity are not responsible for the centralised storage and management of wealth. Further, the value of an unskilled labour-hour in a particular geographic region naturally adjusts according to economic conditions such as inflation, thereby maintaining the long-term intrinsic value of the cryptocurrency. 

This paper is organised as follows: In Section \ref{sec:chronobank-system} we provide an overview of the system as a whole before discussing the technical details of the necessary system components and processes. Section \ref{sec:economic-considerations} provides economic considerations in brief, regarding the real-world deployment of this system and its feasibility. Finally, Section \ref{sec:future-work} discusses future directions and applications of the system and of ChronoBank.
The appendix of this document provides supporting reference of several concepts introduced throughout the paper. In particular Appendix \ref{sec:appendix:minting} and \ref{sec:appendix:fees} list the variables that encapsulate the functionality of the system. These variables are also summarised in Table \ref{tab:variables}. 

\begin{figure*}[t]
	\centering
	\def\svgwidth{\textwidth}
	\input{vectors/Chronobank-overview-simplified.pdf_tex}
	\caption{An overview of the ChronoBank system. This diagram shows the system's constituents and their interactions. CBE refers to the ChronoBank Entity, LOC to Labour-Offering Companies, LHT to Labour-Hour Tokens and TIME to the token purchased during the crowdsale.}
	\label{fig:overview}
\end{figure*}

\section{The ChronoBank System}
\label{sec:chronobank-system}

Similar to existing \textit{stable-coins} (such as USDT by Tether and Digix), we propose a centralised entity that coordinates the creation, redemption, and destruction of Labour-Hour Tokens (LHT). We refer to this entity as the ChronoBank Entity (CBE).  It is responsible for the acquisition and coordination of legally binding contracts for labour, in addition to the creation and dissemination of LHT. Ultimately the role of the CBE is to ensure the stability of the LHT system through careful management of the system's underlying processes. This section will provide details describing the proposed processes, practices, and operations undertaken by the CBE and its associates. An overview of the functionality of the ChronoBank system's constituents is shown in Figure \ref{fig:overview}.


The system as a whole is designed with the intent of a single deployment per economic region. For instance, the system could be deployed once in Australia using the value of one labour-hour in the Australian economy, measured in Australian dollars. As this document is an abstract description of the system, it does not refer to any region-specific implementation but instead refers to generalised system parameters that must be tailored for each region. With few exceptions, all processes and structures described in this document may have slight variations in implementation between regions in which ChronoBank operates. 

The initial implementation of the CBE will utilise the Ethereum\cite{Wood2014} blockchain; however, future implementations may tokenise assets on alternative blockchain systems (e.g. Waves \cite{Ivanov2016}, Bitcoin \cite{Nakamoto2008}) when it is deemed appropriate.

The CBE provides economic benefit both to the environment in which it is deployed, and to a group of stakeholders that assist the CBE by providing initial operating capital. Unlike the region-specific deployment of the CBE system, only one group of CBE stakeholders (a.k.a. TIME token holders) exists globally. These stakeholders provide initial operating capital to the CBE in the form of both fiat and cryptocurrencies, in exchange for a portion of future profits attained by the CBE.


\subsection{Stakeholders \& TIME Tokens}
\label{sec:TIME-tokens}

In order to fund the development and operation of the ChronoBank system, there will be a fundraising phase known as the crowdsale. During the crowdsale, individuals may purchase TIME tokens at a fixed rate, which provides the token holder with the right to receive a proportion of the profits arising from the operation of the system. In addition to the entitlements regarding the system's operating profit, holders of TIME tokens will also be granted the right to vote on important decisions regarding the ChronoBank system.

TIME tokens will be developed utilising the Ethereum ecosystem, specifically leveraging the ERC20 token standard\cite{ERC20}. The ERC20 specification will be extended to provide voting functionality and rewards distribution; this is discussed further in Sections \ref{sec:TIME-voting} and \ref{sec:TIME-rewards} below.

\subsubsection{Crowdsale}
\label{sec:Crowdsale}
During the crowdsale, TIME tokens will be created as necessary and sold at the fixed price of \num{100} TIME tokens for 1 bitcoin (BTC). There is no limit to the number of TIME tokens generated during the crowdsale; however, no further TIME tokens will be generated after this phase of the project.

\begin{table}[H]
  \setlength{\tabcolsep}{8pt}
  \begin{tabular}{c | l}
  $\rho$ & Total percentage of minted LHT taken by the CBE \\
  $f_c$ & Fee taken during minting by the CBE \\
  $f_r$ & Fee taken during redemption by the CBE \\
  $f_i$ & Issuance fee taken during minting \\
  $S$ & Percentage of minted LHT put into the SGF\\
  $L_T$ & Percentage of minted LHT put into the Liquidity Reserve\\
  $L_0$ & Percentage of minted LHT used for LOC insurance\\
  $M$ & Number of months until $L_0$ is transferred into the SGF\\
  $l$ & Percentage of $L_T$ used for LOC insurance\\
  $P$ & Interval between TIME token reward payouts\\
  \end{tabular}
  \caption{List of abstract variables and constants used throughout this paper for reader's reference. See the Appendix for further details. }
  \label{tab:variables}
\end{table}

All TIME tokens purchased during the crowdsale will constitute \num{88}\% of the total TIME tokens generated during the initialisation of the ChronoBank system. The remaining \num{12}\% of tokens will be split with 10\% given to the ChronoBank.io team (for ongoing research and development) and 2\% to advisors and early adopters of the system.

\begin{figure}[H]
	\centering
	\def\svgwidth{\columnwidth}
	\input{vectors/Chronobank-TIME-payout-overview.pdf_tex}
	\caption{Issuance fees ($f_i$) and transaction fees ($f_t$) are deposited into the \textit{rewards contract}. TIME holders can withdraw a portion of the LHT after each snapshot, if their TIME tokens were deposited into the contract.}
	\label{fig:time-payout}
\end{figure}


\subsubsection{TIME Token Rewards}
\label{sec:TIME-rewards}

For the use of the LHT, users will be charged a fixed fee, $f_t = \num{0.15}$\%  on all transactions. Complementary to transaction fees, issuance fees ($f_i$) will be taken during the minting process (further details in Section \ref{sec:minting}). The issuance fees will start at 3\% during the first year of operation and will stepwise decrease by 1\% each year until a final issuance fee of 1\% is achieved and maintained. 

Both the transaction and issuance fees will be automatically collected and sent to a \textit{rewards contract} on the Ethereum blockchain as shown in Figure \ref{fig:time-payout}. The \textit{rewards contract} is designed such that TIME token holders can withdraw their earned rewards in a manner that makes them non-unique (i.e. withdrawal of rewards is not tracked on a per-token basis) and hence tradeable on exchanges for equal value.

The \textit{rewards contract} will allow TIME token holders to retrieve their rewards at regular payout intervals, $P$. At any given stage, TIME holders may deposit their tokens into the \textit{rewards contract}. At the onset of a payout interval, a snapshot of deposited TIME tokens and the current contract reward balance will be taken\footnote{The transaction for which may be initiated by anyone, but typically by the CBE.}. The rewards will be divided equally among the TIME tokens that were deposited at the time of the snapshot. Depositors may then withdraw their share of rewards during the following payout interval. At the next payout snapshot, any unclaimed rewards will be forfeited and added to the total balance of that snapshot. Figure \ref{fig:time-withdraw} illustrates this concept by plotting the \textit{rewards contract} balance over time. We also note that we expect $P$ to be of the order of a few months.  

TIME holders may deposit and withdraw their TIME tokens at any stage, however only TIME holders who have deposited before the snapshot of each payout interval can claim rewards. Withdrawing TIME tokens from the contract will also withdraw any rewards currently owed to the depositor. TIME holders may also leave their tokens deposited in the \textit{rewards contract} indefinitely and claim their rewards periodically. 
\begin{figure}[H]
	\centering
	\def\svgwidth{\columnwidth}
	\input{vectors/Chronobank-time-withdraw.pdf_tex}
	\caption{The \textit{rewards contract} receives $P_r$ rewards for every payout interval $P$ (in our example, we assume that $P_r$ is constant). At each snapshot, the portion of rewards that are not withdrawn is $d$. Rewards that are not withdrawn roll over to the following payout interval $P$. The dashed line indicates the withdrawable balance, which reduce over each payout interval as TIME holders withdraw their rewards.}
	\label{fig:time-withdraw}
\end{figure}
\vspace{-0.2cm}
This reward payout system gives upper bounds to the amount of rewards that can be stored in the contract in any interval, under some reasonable assumptions. The total rewards obtained via fees for a particular payout interval, $P$, is $P_r$. If we assume a constant amount of transaction and issuance fees per payout interval (i.e. a constant $P_r$), and an average percentage of locked TIME holders that do not withdraw in any payout interval, $d$, we can calculate the upper bound of the LHT stored in the smart contract through the limiting geometric series relation\footnote{Assuming $0 \le d <1$.}:
\begin{equation}
  \sum_{k=0}^{\infty} P_r d^k = P_r (1-d)^{-1} ~.
  \label{eq:upperbound}
\end{equation}
To clarify this, let us take a conservative estimate, that  90\% of all locked TIME token holders do not withdraw their rewards in every payout interval (only 10\% actually withdraw). In this scenario, no more than $10 P_r$ worth of rewards will be stored in the \textit{rewards contract} at any given time. A less conservative estimate, being 50\% of locked TIME holders withdrawing each payout interval, will ensure the \textit{rewards contract} never contains more than $2P_r$ worth of rewards. Through the careful (and potentially dynamic) choice of $P$ we can find a balance between practicality (frequency of reward withdrawal) and security (safe level of stored value in a smart contract).  

\subsubsection{TIME Voting}
\label{sec:TIME-voting}
From time to time, the CBE may hold polls on the Ethereum blockchain to elicit the opinion of TIME token holders.  Poll results will be incorporated into decisions made by the CBE concerning the financial or technical direction and/or implementation of the CBE system. Only valid TIME holders are considered current stakeholders in the CBE system and are the only parties authorised to participate in a poll.

\subsection{Labour-Hour Tokens}

Labour-Hour Tokens (LHT) are the fundamental unit of value within the ChronoBank system. The purpose of these tokens is to provide a non-volatile, inflationary-resistant digital store of value on various blockchains. We envisage these tokens for utilisation in future systems, such as LaborX (see Section \ref{sec:future-work} for a brief overview). 

A Labour-Hour Token will be derived from a standard Ethereum ERC20 token and will be tradeable on all major exchanges. The ChronoBank system will ensure that these tokens will always have a 1 to 1 mapping with legally bound promises of labour-hours from various Labour-Offering Companies (LOC). As such, token holders may also redeem these tokens at any given time for their real-world labour-hour counterpart. This section details the various processes required to feasibly sustain the 1 to 1 mapping of LHT to offered labour-hours and ensure the economic stability of the ChronoBank system.  

\subsubsection{Minting}
\label{sec:minting}
The minting process takes place when a company offering labour-hours (Labour-Offering Company (LOC)) chooses to enter into a legally binding agreement with the CBE. The CBE must then run strict checks (the guidelines of which are described in the business outline \cite{BisOutline}) of the LOC wishing to participate in the ChronoBank system. Once the LOC is approved, the CBE and LOC will negotiate on various parameters (summarised in Appendix \ref{sec:appendix:minting}) before entering a legally binding contract whereby the LOC promises labour-hours in exchange for LHT (or the fiat equivalent). The CBE will then publish a hash of the contract on the Ethereum blockchain and store the contract on a decentralised storage system such as IPFS \cite{ipfs} or SWARM \cite{swarm}. This gives a transparent ledger detailing the current state of the backed LHT.  The exact mechanisms behind the minting process are non-trivial and we refer the reader to Figures \ref{fig:mint-sequence} and \ref{fig:mint-distribution} during the following discussion of the process. 
When an LOC and the CBE have agreed on terms, the CBE mints the equivalent amount of LHT as labour-hours offered by the LOC, ensuring the 1 to 1 relationship between the two. Of the newly minted tokens, the CBE will retain $\rho$ percent\footnote{All percentages introduced in this document should be read as decimals. Hence $(1- \rho) \ge 0$.}, providing the remaining $(1 - \rho)$ of minted LHT to the LOC. In practice, the CBE can sell the $(1-\rho)$ of minted LHT on exchanges and provide the LOC with the fiat equivalent, should they not wish to deal with cryptocurrencies. The $\rho$ percent held by the CBE will be immediately subdivided into the following portions (shown diagrammatically in Figure \ref{fig:mint-distribution}). 
\begin{itemize}
  \setlength\itemsep{2mm}
  \item $f_c \in [0,0.01]$ - A fee charged by the CBE for services provided. 
  \item $f_i$ - The issuance fee which will go to the rewards contract for TIME token holders (see Section \ref{sec:TIME-rewards}).
  \item $S$ - A portion to be sent to the Security Guarantee Fund (SGF) (see Section \ref{sec:insurance}). 
  \item $L_{T}$ - The total portion sent to the Liquidity Reserve (Section \ref{sec:liquidity}). This fund is further split by a variable percentage, $l$, into $L_I$ (LOC insurance) and $L_0$ (liquidity) (see Section \ref{sec:liquidity} for further details). 
\end{itemize}
For clarity, we write the obvious explicit relation,
\begin{equation}
  \rho = f_c + f_i + S + L_T ~.
  \label{eq:rho}
\end{equation}
We expect that the system will maintain fixed fees, but will vary $S$, $l$ and $L_T$ (and hence $\rho$) on a case-by-case basis to control the stability and viability of the ChronoBank ecosystem. The purpose of the Liquidity Reserve and Security Guarantee Funds are detailed in the \textit{Funds} Section (\ref{sec:funds}) and a brief discussion of the economic feasibility of this system is discussed in Section \ref{sec:economic-considerations}.
\begin{figure}[H]
	\def\svgwidth{\columnwidth}
	\input{vectors/Chronobank-minting-sequence.pdf_tex}
	\caption{Procedural overview of the ChronoBank minting process.}
	\label{fig:mint-sequence}
\end{figure}

\begin{figure}[H]
	\def\svgwidth{\columnwidth}
	\input{vectors/Chronobank-minting-distribution.pdf_tex}
	\caption{Fund distribution of minted LHT.}
	\label{fig:mint-distribution}
\end{figure}

\subsubsection{Redemption}

As LHT are fundamentally backed by real-world labour-hours, holders may redeem their tokens at any time for these backed hours. To do so, holders will deposit their LHT  into a redemption contract on the Ethereum blockchain. Along with the deposit, holders will provide relevant data detailing their redemption request, such as labour type(s), location(s) of work, date/time of work, contact information, etc. The request will trigger the CBE to search and match relevant LOCs to the redemption request.  If found, the CBE will return the most suitable LOCs that fit the request. To discourage recurrent requests and to remunerate the CBE for its service, the CBE will take a small fee, $f_r \in [0,0.01]$ from the deposited tokens. The amount of labour-hours matched will be less than or equal to the LHT tokens deposited, less the CBE's fee. Any unmatched requests or unfulfilled hours can be withdrawn from the redemption contract by the depositor. As a redundancy, the redemption contract will have a built-in holding period, after which depositors may withdraw their deposited LHT. This protects depositors from inactive or stagnant CBEs. 

Once the CBE has returned a list of compatible LOCs, a depositor may accept or reject the offered LOC(s). If rejected, the depositor can withdraw their deposited LHT less the CBE's service fee. If accepted, the CBE will advise the chosen LOC of all relevant details. The redemption contract will hold the LHT until the work has been completed and the depositor is satisfied. A dispute resolution system may be used to ensure work is completed as requested. Once the labour has been satisfactorily completed, the deposited LHT will be destroyed, again ensuring the 1 to 1 relationship of LHT to backed labour-hours.  

\subsubsection{Intrinsic Value}
\label{sec:intrinsic-value}

We have thus far ensured a 1 to 1 correspondence between LHT and labour-hours but have not yet defined the value of either. The intrinsic value of one LHT and thus one labour-hour is region-dependant. Although any arbitrary value can be chosen, we peg the value of one LHT to be roughly the average hourly wage of a region for practical reasons. 

The average hourly wage of a region will be defined via the official statistics bureau of that specific region. If one were to trivially adjust the LHT to the most recent statistics released for a given region, the price of LHT would stepwise shift on the release date of the statistics. Typically these statistics are released yearly and in such scenarios a predictive stepwise jump each year would occur. To implement a smooth transition from one statistics point to another, we propose that the pegged price of LHT be a linear interpolation between the two points. This would require the LHT price to lag behind the most recent statistics by the duration of at least one extra statistics release.

Let us illustrate this with a clear example, as depicted in Figure \ref{fig:pegged-price}. Let us determine the price of LHT during April, 2010 in a region where the statistics bureau publishes a data point in January each year. This data point indicates the average wage in that region for the previous year. The price will be calculated by retrieving two data points: $A_{2008}$ and $A_{2009}$, the average wage in 2008 (released in January, 2009) and the average wage in 2009 (released in January, 2010) respectively. We take a linear interpolation between these two points to find the price of LHT in April, 2010. In this example, this will over-approximate\footnote{This is a poor approximation that assumes a linear growth, where the start of the year is the average wage of that year and the end of the year the is average wage of the following year.} the average wage in April, 2008. This example shows that the LHT price in April 2010 will be the over-approximated average wage in April 2008, demonstrating how the pegged price of LHT will lag behind the current actual average wage. It should be noted that this doesn't affect any aspects of offering or redeeming labour. The value of labour offered or redeemed is independent of the price of a single LHT. However, the value will be measured with respect to the fixed price of a single LHT token.  

\begin{figure}[H]
	\centering
	\def\svgwidth{\columnwidth}
	\input{vectors/Chronobank-price-pegging.pdf_tex}
	\caption{The base price of one LHT in April 2010, $X$, can be calculated as the linear interpolation between the average wage in 2008 and 2009 ($A_{2008}$ and $A_{2009}$).}
	\label{fig:pegged-price}
\end{figure}

For any given region, typically there are sub-regions which have varying costs associated with providing labour work. We wish to integrate these costs within our definition of a single LHT. To do so, we take the maximum of each cost within a sub-region and add it to the average wage of the region to provide the fundamental price of LHT. Specifically we have: 
\begin{equation}
  \mathcal{L} = \left(1 + \max(Y)  \right) X ~,
  \label{eq:LHT}
\end{equation}
where X is the linearly interpolated function of average wages described above and $Y$ is the individual cost in a sub-region as a percentage of work done. $\mathcal{L}$ as defined in equation (\ref{eq:LHT}) is the pegged price of LHT for any given region. This value will vary linearly throughout each year in a transparent and predictable manner. 


\subsection{Funds}
\label{sec:funds}

The uniqueness of backing a digital token with contractual debt requires various safeguards to ensure contracts are always upheld and potential defaulting is accounted for. There are a number of adverse scenarios that can occur with a debt-backed system. Our proposed solutions involve the careful maintenance of two extra funds, the Liquidity Reserve and the Security Guarantee Fund (SGF). In this section we detail the operation of these funds and how they address some of the issues that can arise in this system. 

\subsubsection{Liquidity Reserve}
\label{sec:liquidity}

The Liquidity Reserve is an offline LHT storage fund controlled by the CBE. It receives a percentage, $L_T$, of newly minted LHT during the minting process (Section \ref{sec:minting}). There are two services that this function provides the ChronoBank system: 

\begin{enumerate}
	\item \textit{LOC Risk Mitigation} - During the minting process, an LOC will be paid $(1-\rho)$ percent of the labour-hours they contractually agree to provide. Therefore LOCs inherently take some risk in the form of immediate redemption. If an LOC's promise of labour-hours is immediately redeemed, they stand to lose $\rho$ percent of these hours. To mitigate this, the CBE will store a percentage of minted LHT, $L_I$ (see Section \ref{sec:minting}) for each LOC, to reimburse the LOC in the event that more than $(1-\rho)$ labour-hours are redeemed. $L_I$ will cover all excess (i.e. more than $(1-\rho)$ ) redeemed tokens until the reserve, $L_I$, is depleted. 

We formalise this by introducing the mitigated tokens payed to an LOC, $L_M$, as
\begin{equation}
	L_M = \min(L_I,E) ~.
  \label{eq:redeemed}
\end{equation}
where $E$ is the excess LHT to be redeemed by the LOC, defined as: 
\begin{equation}
  E = \begin{cases} 
    R - (1 -\rho) , & \text{if $R > (1 - \rho)$}. \\
    0,  & \text{if $R \le (1- \rho)$}. 
  \end{cases}
  \label{eq:E}
\end{equation}
with $R$ being the total percentage of LHT redeemed.

  The mitigation reserve, $L_I$ is not constant, but is gradually transferred to the SGF (Section \ref{sec:insurance}). The rate at which this fund is transferred is derived from the parameter, $M$, which specifies the total number of months until the funds are entirely transferred, and is negotiated during the minting process. The mitigation reserve funds will be transferred monthly at a uniform rate ($L_I/M$).  

  This procedure then allows the CBE to statistically choose $\rho$, $L_T$ and $M$ given the risk profile and reputation of any given LOC during the minting process. For example, $\rho$, $L_T$ and $M$ can be designed such that within a 95 percent confidence interval, an LOC does not lose more than $(\rho - \zeta)$ worth of the labour-hours promised over a given time\footnote{This can occur because the holdings of an LOC increase over time due to assumed external investment.}. Here $\zeta$ is a number that quantifies the risk that the LOC is willing to accept. This degree of freedom allows the CBE to manage LOCs of varying risk profiles. 

\item \textit{LHT Liquidity} - The price of LHT will ultimately be governed by the price at which the CBE buys and sells LHT\footnote{Combined with the fact that one LHT has a fundamental intrinsic value of one labour-hour.}. The funds used for this are stored in the Liquidity Reserve. $L_0$ of newly minted tokens (see Section \ref{sec:minting}) will be accrued in this fund. The CBE will then stabilise the price of LHT and provide liquidity in various markets by buying and selling LHT at the fundamental price detailed in Section \ref{sec:intrinsic-value}. 

The parameter, $l$, chosen during contract negotiations, specifies the percentage of $L_T$ that goes to $L_I$ (the amount used for LOC insurance and which is gradually transferred to the Security Guarantee Fund) and $L_0$ (the amount that is permanently used for liquidity). Through careful management of the parameter, $l$, the CBE can maintain the desired volume of funds in the Liquidity Reserve. As this fund will hold both LHT and volatile currencies, care must be taken to manage volume due to the potential price fluctuations of the held currencies. 

One immediate-use case of this fund will be in the initial system set-up. When the first LOCs become part of the system, the freshly minted LHTs will need to be sold to transfer the resulting funds to the LOC. Initially the demand for LHT will be low and funds from the Liquidity Reserve will be used to bootstrap this process. A percentage of the crowdsale funds will be deposited into the Liquidity Reserve for this purpose.
\end{enumerate}


\subsubsection{Security Guarantee Fund}
\label{sec:insurance}

One of the major drawbacks in a debt-backed currency system is the possibility of backers (LOC in our case) defaulting on their contractual obligations. Despite the care in vetting LOCs during the minting process, we expect there to be a percentage of companies that will inevitably default. The SGF's main purpose is to provide a fund reservoir as insurance to protect against defaulting companies. In practice, this fund will burn held LHT tokens to the equivalent amount of outstanding labour-hours promised by the defaulting LOC.  

Statistical estimates should be taken for the average probability of LOCs defaulting. These estimates will give a measure defining the amount of LHT that should be stored inside the SGF fund at any given time. The amount stored in the SGF will be proportional to the amount of debt, and hence be proportional to the amount of LHT in existence (as every LHT is backed by one owed labour-hour). This required value can be reached and maintained through the management of the minting variables, $S$, $l$ and $L_T$. 

\section{Economic Considerations}
\label{sec:economic-considerations}

In this section we briefly mention some interesting properties and immediate economic consequences of this system.


\subsection{Economic Incentives}
First and foremost, the system must be designed such that there are economic incentives for both LOCs and LHT holders to participate in the system. For an LOC, the incentive to participate in the ChronoBank system comes in the form of an interest-free-like loan. When an LOC agrees to participate in the system and offers labour-hours, the company effectively receives an interest-free loan, which needs to be paid back when their contract expires. For this to be enticing to an LOC, we require that over the period of the contract the amount of money paid for the service of the loan is less than that of alternate means for loans, e.g. local bank loans. 

We consider a simplistic example to demonstrate the feasibility of such a scenario. Consider that a bank loan, which charges $I_B$ percent in interest per annum, has an upfront cost of $C$ and is of total amount $H$. Over a period of time, $t$, and assuming no regular repayments are necessary\footnote{We use a no-repayment loan to clarify the analogy to our system and remove some mathematical complexity which obscures our point.}, the LOC stands to lose: 
\begin{equation}
  C + H ( (1 + I_B)^t -1) - H I_I t ~.
\end{equation}
The first term corresponds to the upfront cost of the loan, the second term to total interest owed at time, $t$, and the final term is potential interest earnings from external investments. Here we've defined an average yearly interest gain from external investments, $I_I$. 

Alternatively, the LOC may participate with the ChronoBank system. If the LOC offers $H$ hours of labour, it would receive $H(1-\rho)$ LHT in return, and would be required to pay $(1 + \sigma t)H$ worth of LHT back upon the expiration of the contract (assumed at time, $t$). Here $\sigma$ is a percentage representing the average yearly price increase\footnote{This could also potentially decrease.} of LHT. It quantifies an inflated price of LHT due to an assumed increase in average wage over the contract period. Therefore, with the ChronoBank system, the LOC stands to lose, 
\begin{equation}
  H(\rho + \sigma t) - (1-\rho) H I_I t ~. 
\end{equation}
The first term here originates from the initial sum taken by ChronoBank, $H\rho$, and from the price fluctuation of LHT over time. The second term comes from the potential earnings from external investments with the $(1 -\rho)H$ LHT received. 

In this simplistic example, an LOC would be economically incentivised to participate in the ChronoBank system if the total fees of the ChronoBank system are less than the total fees of the alternative bank loan, explicitly:
\begin{equation}
  \begin{aligned}
%& H(\rho  + \sigma t) - (1- \rho) H I_I t  < C + H \left( (1 + I_B)^t - 1 \right) - H I_I t \\
%&\Rightarrow 
H( \rho + (\sigma + \rho I_I) t) < C + H \left( (1 + I_B)^t -1 \right) ~.
\end{aligned}
  \label{eq:incentivised}
\end{equation}
Here, $(\sigma + \rho I_I)$ represents the average yearly price fluctuation of LHT combined with $\rho$ percent of an average expected investment return. With the reasonable assumption that this quantity is less than the interest rate of a bank loan, $I_B$, equation (\ref{eq:incentivised}) can always be satisfied for some time period, $t$, given all other variables fixed. In fact, the longer the contractual time, $t$, the greater the cost-saving is to the LOC. So in this simplistic example, we can see that this system incentivises LOCs to participate in the ChronoBank system for longer periods of time\footnote{Adding regular repayments to bank loans adds complexity to this simplistic example and although decreasing the size of the right-hand side of equation (\ref{eq:incentivised}) it does not change our final result.}. 

Furthermore, the CBE and LOC are able to adjust both $H$ and $\rho$ during the negotiations of their initial contractual agreement, enabling the tuning of economic incentives in less favourable economic regions/scenarios. 

As for LHT holders, their economic incentive is more obvious. LHT, compared to its \textit{stable-coin} predecessors, is inflationary-resistant. Therefore, holders should have an economic incentive to use LHT as its value should increase relative to local inflationary fiat currencies. We should note that buying and holding LHT as an investment (and therefore decreasing the liquidity of the token) is not in a holder's best interest, as the gains in doing so are often less than external investment strategies.  

\subsection{System Stability}

This section is concerned with the ability of the ChronoBank system to sustain various economic hurdles, which we refer to as its stability.

The stability of the ChronoBank system hinges on the CBE correctly managing the minting process. Through the minting process, the two funds, Liquidity Reserve and SGF, are controlled and maintained. 

As the system grows, funds will accrue in both the Liquidity Reserve and the SGF. LHT is only removed from the SGF in the event of an LOC defaulting. The Liquidity Reserve only decreases in value in the event that a held volatile currency devalues. Therefore, both funds should grow as the system evolves. As the funds reach a sustainable level, the percentage of LHT taken during the minting, $\rho$, can be decreased, further enhancing the economic incentive for more LOCs to join the system. A greater number of LOCs participating will mean a greater number of LHT in existence and a greater volume of funds accrued in both the SGF and Liquidity Reserve. The stability of the system will be proportional to the value stored in the SGF and Liquidity Reserve and, therefore, we expect the system to become more stable as it develops. 

\subsection{Potential Pitfalls}

A number of potential pitfalls can occur during the operation of this system. In this section we briefly summarise the major and most obvious ones, along with our proposed solutions. 
\begin{itemize}
  \setlength\itemsep{2mm}
  \item An LOC defaulting on its promise of labour-hours. As previously mentioned, this will be covered by the SGF. The CBE will burn the necessary LHT from the SGF fund to maintain the 1 to 1 relation of LHT to labour-hours.
  \item Large supply/demand causing price fluctuations. This can be handled by a sufficiently deep Liquidity Reserve, which will provide demand in times of high supply and vice versa.   Initially funds from the crowdsale will be placed into the Liquidity Reserve, and as the system grows the Liquidity Reserve will be maintained at a level deemed operationally safe to cover this scenario. 
  \item Redemption of all LHT. The LHT at any given time will always be backed by contractual labour-hours in a 1 to 1 mapping. Therefore, this scenario is possible and the system will continue to function in this event. However, the risk in this occurrence lies with the participating LOCs, who will be required to pay back their LHT in the form of labour. The SGF (Section \ref{sec:liquidity}) can also absorb some of the cost of this scenario.  It will be at the CBE's discretion to use the SGF to assist in this very unlikely scenario.  
  \item Increased demand at contract expiry. As a contract expires, an LOC will be required to buy back an equivalent amount of LHT to the labour-hours that are left on the contract. This could potentially create periods of large demand. In these scenarios, we will counter the demand with the Liquidity Reserve and, if necessary, the SGF. 
\end{itemize}


\section{Future Work}
\label{sec:future-work}

\paragraph{Economic Models}
Key to the success of the ChronoBank system is an informed choice of the aforementioned economic parameters. It is essential to perform further analysis so as to determine how parameter modification impacts the feasibility and sustainability of the system in a wider context. This will necessarily be performed before a real-world implementation is constructed. 

\paragraph{LaborX}
The digital asset transfer system described in this document is proposed as a first step towards a larger decentralised labour system. LHT as described by this paper forms the base currency for a decentralised labour exchange platform entitled LaborX. The intention of LaborX is to enable peer-to-peer exchange of labour-hours with LHT, thereby reducing the centralisation of the proposed ChronoBank system. LaborX will incorporate a rating system whereby holders of LHT can identify fair trades by examining the quality and/or specialisation of the labour provider, given their history on the platform. By enabling direct exchange of LHT with labour-hours, the system's dependency on contractual arrangements with LOCs is significantly reduced. This potentially reduces the cost and increases the stability of the system as a whole.

\section{Conclusion}

This paper proposes a non-volatile, inflationary-resistant, digital asset transfer system. This innovative system is only made possible by recent advancements in blockchain and cryptographic technologies. Leveraging these technologies, this system tokenises contractual debt in a manner that can be both economically feasible and highly practical for digital platforms, such as LaborX. The intrinsic value of the proposed token mirrors the average hourly rate of human labour - the most fundamental unit of economic value.

\bibliographystyle{unsrt}
\bibliography{references}

\end{multicols}
\pagebreak
\appendix

\section{Terminology}
\label{sec:appendix:terminology}

The following is a list of terms used throughout this document.
\begin{itemize}
  \setlength\itemsep{2mm}
  \item \textbf{CBE} - The ChronoBank Entity.
  \item \textbf{LHT} - Labour-Hour Token.
  \item \textbf{LOC} - Labour-Offering Company.
  \item \textbf{SGF} - Security Guarantee Fund.
\end{itemize}

\section{Minting Contract Parameters}
\label{sec:appendix:minting}

There are number of parameters that are required to be negotiated during the minting process. For clarity we list here the variables that should be considered and determined by the CBE during the minting process: 

\begin{itemize}
  \setlength\itemsep{2mm}
  \item $S$ - A percentage of the minted tokens to be stored in the Security Guarantee Fund.
  \item $L_T$ - The total percentage to be stored in the Liquidity Reserve. 
  \item $l$ - The percentage of $L_T$ dictating how much of the Liquidity Reserve is used as LOC insurance, $L_I$. The remaining funds will be kept for liquidity in $L_0$. 
  \item $M$ - The total number of months until the LOC Insurance $L_I$ is transferred to the Security Guarantee Fund. This dictates the rate at which the funds are transferred, i.e. $L_I/M$ per month. 
  \item Expiry Date - The expiry date of the contract, whereby the LOC must buy back the value of the minted LHT. 
\end{itemize}
Through the definition of the above variables, the CBE on a case-by-case basis will fix $\rho$, the total percentage deducted from LOCs initially through the relation (\ref{eq:rho}). \\


\section{Fee Summary}
\label{sec:appendix:fees}

The fees of the ChronoBank system can be summarised in the following:
\begin{itemize}
  \setlength\itemsep{2mm}
  \item $f_c$ - This is the fee taken by the CBE during the minting process for services rendered. We expect $f_c \in [0, 0.01]$. 
  \item $f_i$ - This is the issuance fee, which will be taken during minting and deposited into the TIME token holder's \textit{rewards contract}. This will start at 3 percent and decrease by 1 percent each year until it is maintained at a steady 1 percent. 
  \item $f_t$ - These are transaction fees, which are deposited into the TIME token holders \textit{rewards contract}. This is a flat fee of 0.15\%. 
   \item $f_r$ - This is a fee taken by the CBE during the redemption process for services provided and to add a deterrent to continual redemption requests. We expect $f_c \in [0,0.01]$. 
\end{itemize}



\end{document}
